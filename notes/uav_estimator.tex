%!TEX TS-program = xelatex
%!TEX encoding = UTF-8 Unicode

\documentclass[a4paper]{article}
\usepackage[a4paper, margin=2in]{geometry}

\usepackage[hidelinks]{hyperref}
\usepackage{graphicx}
\usepackage{xcolor}
\usepackage{tabularx}
\usepackage{amsmath,amssymb,amsbsy,amsfonts}
\usepackage{accents}
\usepackage{nicefrac}
\usepackage{upgreek}
\usepackage[]{footmisc}

\usepackage{geometry}
\geometry{a4paper, margin=1in}

% !TEX root=root.tex

% vectors, quaternions, etc.
\newcommand*{\vect}[1]{\boldsymbol{\mathrm{#1}}}
\newcommand*{\quat}[1]{\boldsymbol{\mathrm{#1}}}

\newcommand*{\x}{\vect{x}}
\newcommand*{\xt}{\tilde{\vect{x}}}
\newcommand*{\xhat}{\hat{\vect{x}}}
\newcommand*{\xbar}{\bar{\vect{x}}}

\newcommand*{\z}{\vect{z}}
\newcommand*{\zhat}{\hat{\vect{z}}}
\newcommand*{\zbar}{\bar{\vect{z}}}

\newcommand*{\q}{\quat{q}}
\newcommand*{\e}{\vect{e}}
\newcommand*{\qt}{\tilde{\quat{q}}}
\newcommand*{\drag}{c_d}

% Trig functions
\newcommand*{\sphi}{s_\phi}
\newcommand*{\cphi}{c_\phi}
\newcommand*{\stheta}{s_\theta}
\newcommand*{\ctheta}{c_\theta}
\newcommand*{\ttheta}{t_\theta}
\newcommand*{\spsi}{s_\psi}
\newcommand*{\cpsi}{c_\psi}

% modifiers
% \newcommand*{\desired}[1]{\accentset{\circ}{#1}}
\newcommand*{\des}[1]{\check{#1}}

% norms, transpose, etc.
\newcommand*{\transpose}{\mathsf{T}}
\newcommand*{\skewmat}[1]{\left[ #1 \right]_{\times}}
\newcommand*{\norm}[1]{\left\Vert #1\right\Vert }
\newcommand*{\abs}[1]{\left\vert #1\right\vert }

% group/algebra names
\newcommand*{\SO}{\ensuremath{\mathit{SO}}}
\newcommand*{\so}{\ensuremath{\mathfrak{so}}}
\newcommand*{\SE}{\ensuremath{\mathit{SE}}}
\newcommand*{\se}{\ensuremath{\mathfrak{se}}}

% operators
\DeclareMathOperator*{\argmin}{arg\,min}
\DeclareMathOperator*{\Exp}{Exp}
\DeclareMathOperator*{\Log}{Log}

% reference commands
%\renewcommand*{\eqref}[1]{Eq.~\ref{#1}}
\renewcommand*{\eqref}[1]{(\ref{#1})}
\newcommand*{\tabref}[1]{Table~\ref{#1}}
\newcommand*{\secref}[1]{Sec.~\ref{#1}}
\newcommand*{\appxref}[1]{Appx.~\ref{#1}}

% comments
\definecolor{orange}{rgb}{1,0.5,0}
\definecolor{darkgreen}{rgb}{0,0.6,0}
\newcommand{\JJ}[1]{{\color{orange}{JJ: #1}}}
\newcommand{\RB}[1]{{\color{teal}{RB: #1}}}
\newcommand{\TM}[1]{{\color{purple}{TM: #1}}}
\newcommand{\DK}[1]{{\color{darkgreen}{DK: #1}}}
\newcommand{\JN}[1]{{\color{red}{JN: #1}}}
\newcommand{\MF}[1]{{\color{red}{MF: #1}}}
\newcommand{\TODO}{{\color{red}{TODO}}}

% theorem environments
\newtheorem{theorem}{Theorem}[subsection]
\newtheorem{corollary}{Corollary}[theorem]
\newtheorem{lemma}[theorem]{Lemma}



\title{\LARGE \bf UAV Precision Maritime Landing}

\author{Michael Farrell}


\begin{document}
\maketitle

\section{UAV State Estimator}
\subsection{UAV State}
The purpose of this section is to develop a full state estimator for a
multirotor UAV. The state that we want to estimate is given by
\begin{equation}
  \x =
  \begin{bmatrix}
    p_n & p_e & p_d &
    \phi & \theta & \psi &
    u & v & w
  \end{bmatrix}^\transpose.
\label{eq:uav_state}
\end{equation}

\subsection{Sensors}
We assume that the UAV has a sensor suite contatining the following sensors:
\begin{itemize}
  \item GPS (Position, Velocity)
  \item IMU (Accel, Angular rate, RPY angles?)
  \item Altimeter (Baro or Laser)
\end{itemize}

\subsection{Estimator}
As a first pass, we assume that there are no biases on the sensors. We do,
however, assume that there is linear drag acting on the UAV and therefore
we must estimate the drag term. Our estimated state is equal to the UAV state
given in (\ref{eq:uav_state}) appeneded with this drag term, $\mu$,
\begin{equation}
  \x =
  \begin{bmatrix}
    p_n & p_e & p_d &
    \phi & \theta & \psi &
    u & v & w &
    \mu
  \end{bmatrix}^\transpose.
\label{eq:est_state}
\end{equation}
As the inputs to the estimator, we use the IMU measurements, namely:
\begin{equation}
  \vect{u} =
  \begin{bmatrix}
    a_z &
    p & q & r
  \end{bmatrix}^\transpose.
\label{eq:est_inputs}
\end{equation}
Note that we only use the acceleration in the $z$ direction as an input. The
other accelerations, $a_x$ and $a_y$, will be used as measurements with the drag
model.

\subsection{Dynamics}
We use common UAV dynamics with the addition of a linear drag term as explained
in~\cite{leishman2014accel}. The UAV dynamics are given by
\begin{align}
  \begin{bmatrix}
    \dot{p_n} \\
    \dot{p_e} \\
    \dot{p_d}
  \end{bmatrix}
  &=
  R_b^I
  \begin{bmatrix}
    u \\
    v \\
    w
  \end{bmatrix}
  \\
  \begin{bmatrix}
    \dot{\phi} \\
    \dot{\theta} \\
    \dot{\psi}
  \end{bmatrix}
  &=
  \begin{bmatrix}
    1 & \sin\phi\tan\theta & \cos\phi\tan\theta \\
    0 & \cos\phi & -\sin\phi \\
    0 & \frac{\sin\phi}{\cos\theta} & \frac{\cos\phi}{\cos\theta}
  \end{bmatrix}
  \begin{bmatrix}
    p \\
    q \\
    r 
  \end{bmatrix}
  \\
  \begin{bmatrix}
    \dot{u} \\
    \dot{v} \\
    \dot{w}
  \end{bmatrix}
  &=
  R_I^b
  \begin{bmatrix}
    0 \\
    0 \\
    g
  \end{bmatrix}
  +
  \begin{bmatrix}
    vr - wq \\
    wp - ur \\
    uq - vp
  \end{bmatrix}
  -
  \begin{bmatrix}
    0 \\
    0 \\
    a_z
  \end{bmatrix}
  -
  \begin{bmatrix}
    \mu u \\
    \mu v \\
    0
  \end{bmatrix}
  \\
  \dot{\mu} &= 0.
\end{align}
It is also imporant to note that the rotation from the inertial frame to the
body frame, $R_I^b$, is given by
\begin{equation}
  R_I^b =
  \begin{bmatrix}
    \ctheta \cpsi & \ctheta \spsi & -\stheta \\
    \sphi \stheta \cpsi - \cphi \spsi & \sphi \stheta \spsi + \cphi
    \cpsi & \sphi \ctheta \\
    \cphi \stheta \cpsi + \sphi \spsi & \cphi \stheta \spsi - \sphi
    \cpsi & \cphi \ctheta
  \end{bmatrix}
\end{equation}
and the rotation from the body frame to the inertial frame is given by
\begin{equation}
  R_b^I = \left(R_I^b\right)^\transpose.
\end{equation}
To make it easier to derive the jacobians in the following section, we expand
the dynamic equations.
\begin{align}
  \dot{p_n} &= (\ctheta \cpsi)u + (\sphi \stheta \cpsi - \cphi \spsi)v +
    (\cphi \stheta \cpsi + \sphi \spsi)w \\
  \dot{p_e} &= (\ctheta \spsi)u + (\sphi \stheta \spsi + \cphi \cpsi)v +
    (\cphi \stheta \spsi - \sphi \cpsi)w \\
  \dot{p_d} &= (-\stheta)u + (\sphi \ctheta)v + (\cphi \ctheta)w \\
  \dot{\phi} &= p + \sphi \ttheta q + \cphi \ttheta r \\
  \dot{\theta} &= \cphi q - \sphi r \\
  \dot{\psi} &= \frac{\sphi}{\ctheta} q + \frac{\cphi}{\ctheta} r \\
  \dot{u} &= -\stheta g + vr - wq - \mu u \\
  \dot{v} &= \sphi \ctheta g + wp - ur - \mu v \\
  \dot{w} &= \cphi \ctheta g + uq - vp - a_z \\
  \dot{\mu} &= 0
\end{align}

\subsection{Motion Model Jacobians}
For use in the EKF, we need the jacobians of the dynamics. We need both the
jacobian with respect to the state,
\begin{equation}
  A = \frac{d f\left(\x, \vect{u}\right)}{d \x}
\end{equation}
as well as the jacobian with respect to the inputs,
\begin{equation}
  B = \frac{d f\left(\x, \vect{u}\right)}{d \vect{u}}
\end{equation}

\subsubsection{State Jacobians}
\begin{equation}
  A =
  \begin{bmatrix}
    \vect{0} & \frac{\partial \dot{\vect{p}}}{\partial \vect{\theta}} &
    \frac{\partial \dot{\vect{p}}}{\partial \vect{v}} & \vect{0} \\
    \vect{0} & \frac{\partial \dot{\vect{\theta}}}{\partial \vect{\theta}} &
    \vect{0} & \vect{0} \\
    \vect{0} & \frac{\partial \dot{\vect{v}}}{\partial \vect{\theta}} & \frac{\partial \dot{\vect{v}}}{\partial \vect{v}} & \frac{\partial \dot{\vect{v}}}{\partial \mu} \\
    \vect{0} & \vect{0} & \vect{0} & 0 
  \end{bmatrix}
\end{equation}
The partial jacobians are then given by:
\begin{align*}
  \dot{p_n} &= (\ctheta \cpsi)u + (\sphi \stheta \cpsi - \cphi \spsi)v +
    (\cphi \stheta \cpsi + \sphi \spsi)w \\
  \frac{\partial \dot{p_n}}{\partial \phi} &= (\cphi \stheta \cpsi + \sphi
    \spsi) v + (-\sphi \stheta \cpsi + \cphi \spsi) w \\
  \frac{\partial \dot{p_n}}{\partial \theta} &= (-\stheta \cpsi) u + (\sphi
    \ctheta \cpsi) v + (\cphi \ctheta \cpsi) w \\
  \frac{\partial \dot{p_n}}{\partial \psi} &= (-\ctheta \spsi) u +(-\sphi
    \stheta \spsi - \cphi \cpsi) v + (-\cphi \stheta \spsi + \sphi \cpsi)
    w \\
  \\
  \dot{p_e} &= (\ctheta \spsi)u + (\sphi \stheta \spsi + \cphi \cpsi)v +
    (\cphi \stheta \spsi - \sphi \cpsi)w \\
  \frac{\partial \dot{p_e}}{\partial \phi} &= (\cphi \stheta \spsi - \sphi
    \cpsi) v + (-\sphi \stheta \spsi - \cphi \cpsi) w \\
  \frac{\partial \dot{p_e}}{\partial \theta} &= (-\stheta \cpsi)u + (\sphi
    \ctheta \cpsi)v + (\cphi \ctheta \cpsi)w \\
  \frac{\partial \dot{p_e}}{\partial \psi} &= (\ctheta \cpsi) u + (\sphi \stheta
    \cpsi - \cphi \spsi) v + (\cphi \stheta \cpsi + \sphi \spsi) w \\
  \\
  \dot{p_d} &= (-\stheta)u + (\sphi \ctheta)v + (\cphi \ctheta)w \\
  \frac{\partial \dot{p_d}}{\partial \phi} &= (\cphi \ctheta) v + (-\sphi
    \ctheta) w \\
  \frac{\partial \dot{p_d}}{\partial \theta} &= (-\ctheta) u + (-\sphi \stheta)
    v + (-\cphi \stheta) w \\
  \frac{\partial \dot{p_d}}{\partial \psi} &= 0 \\
  \\
  \frac{\partial \dot{\vect{p}}}{\partial {\vect{v}}} &= R_b^I \\
  \\
  \dot{\phi} &= p + \sphi \ttheta q + \cphi \ttheta r \\
  \frac{\partial \dot{\phi}}{\partial \phi} &= \cphi \ttheta q -
    \sphi \ttheta r \\
  \frac{\partial \dot{\phi}}{\partial \theta} &= \frac{\sphi}{\ctheta \ctheta} q
    + \frac{\cphi}{\ctheta \ctheta} r \\
  \frac{\partial \dot{\phi}}{\partial \psi} &= 0 \\
  \\
  \dot{\theta} &= \cphi q - \sphi r \\
  \frac{\partial \dot{\theta}}{\partial \phi} &= -\sphi q - \cphi r \\
  \frac{\partial \dot{\theta}}{\partial \theta} &= 0 \\
  \frac{\partial \dot{\theta}}{\partial \psi} &= 0 \\
  \\
  \dot{\psi} &= \frac{\sphi}{\ctheta} q + \frac{\cphi}{\ctheta} r \\
  \frac{\partial \dot{\psi}}{\partial \phi} &= \frac{\cphi}{\ctheta} q +
    \frac{-\sphi}{\ctheta} r \\
  \frac{\partial \dot{\psi}}{\partial \theta} &= \frac{\sphi}{\ctheta} \ttheta q +
  \frac{\cphi}{\ctheta}\ttheta r \\
  \frac{\partial \dot{\psi}}{\partial \psi} &= 0 \\
\end{align*}
\TODO

\subsubsection{Input Jacobians}
\begin{equation}
  B =
  \begin{bmatrix}
    0 & \vect{0} \\
    0 & \frac{\partial \dot{\vect{\theta}}}{\partial \vect{\omega}} \\
    \frac{\partial \dot{\vect{v}}}{\partial a_z} & \frac{\partial
      \dot{\vect{v}}}{\partial \vect{\omega}} \\
    0 & \vect{0} 
  \end{bmatrix}
\end{equation}

\begin{align*}
  \begin{bmatrix}
    \dot{\phi} \\
    \dot{\theta} \\
    \dot{\psi}
  \end{bmatrix}
  &=
  \begin{bmatrix}
    1 & \sin\phi\tan\theta & \cos\phi\tan\theta \\
    0 & \cos\phi & -\sin\phi \\
    0 & \frac{\sin\phi}{\cos\theta} & \frac{\cos\phi}{\cos\theta}
  \end{bmatrix}
  \begin{bmatrix}
    p \\
    q \\
    r 
  \end{bmatrix}
  \\
  \frac{\partial \dot{\vect{\theta}}}{\partial \vect{\omega}} &=
  \begin{bmatrix}
    1 & \sin\phi\tan\theta & \cos\phi\tan\theta \\
    0 & \cos\phi & -\sin\phi \\
    0 & \frac{\sin\phi}{\cos\theta} & \frac{\cos\phi}{\cos\theta}
  \end{bmatrix}
\end{align*}
\TODO

\bibliographystyle{IEEEtran}
\bibliography{abbrev,library}
\end{document}
