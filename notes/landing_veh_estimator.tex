% !TEX root=./root.tex

\section{UAV + Landing Vehicle Estimator}
Now assume that in addition to estimating the UAV states, we also want to
estimate the location of a landing target as it moves on the ground. As a first
pass, we assume that the landing target is constrained to move on a plane so
that it only has three degrees of freedom. The only information we receive about
the landing vehicle are visual observations of landmark points and a full
transform from the camera to the landing marker.

Let us clarify some refernce frames.
\begin{itemize}
  \item $I$ - Inertial frame. The inertial frame is an inertially fixed, North,
    East, Down coordinate frame.
  \item $b$ - Body frame. The body frame is a Forward, Right, Down (NED) frame
    that is attached to the UAV. It rotates and moves with the UAV.
  \item $c$ - Camera frame. The camera frame is rigidly attached to the body
    frame at a constant offset and rotation. The $z$ axis of the camera frame
    points out the lens of the camera, the $x$ axis points to the right in the
    camera image, and the $y$ axis down in the camera image.
  \item $v$ - Vehicle frame. The vehicle frame frame is colocated with the body
    frame, but is aligned with the $I$ frame.
  \item $g$ - Goal frame. The goal frame is a Forward, Right, Down (NED) frame
    that is attached to the landing vehicle. It rotates and moves with the UAV.
    The position (0, 0, 0) in the $g$ frame is the desired landing position for
    the UAV.
\end{itemize}

\subsection{State}
The state of the combined system can be separated into the individual parts for
the UAV and the landing goal
\begin{equation}
  \x =
  \begin{bmatrix}
    \left(\x_{\text{UAV}}\right)^\transpose & \left(\x_{\text{Goal}}\right)^\transpose
  \end{bmatrix}^\transpose.
\end{equation}
These individual components are expressed as
\begin{align}
  \x_{\text{UAV}} &=
  \begin{bmatrix}
    p_n & p_e & p_d &
    \phi & \theta & \psi &
    u & v & w
  \end{bmatrix}^\transpose \\
  \x_{\text{Goal}} & =
    \begin{bmatrix}
      \vect{p}_{g/v}^{v} & \rho & \vect{v}_{g/I}^{g} & \theta_{I}^{g} &
      \omega_{g/I}^{g} &
      \vect{r}_{1}^{g} & \rho_{1} & \dots & \vect{r}_{n}^{g} & \rho_{n}
    \end{bmatrix}^{\transpose}.
\end{align}
The UAV states are self explanatory, covering position, attitude and velocity.
The Goal states, however, are a little more unique. They are described here:
\begin{itemize}
  \item $\vect{p}_{g/v}^{v}$ - The position of the goal frame w.r.t. vehicle
    frame, expressed in the vehicle frame. This position does not change as
    either the UAV or the goal rotates because the vehicle frame never rotates.
  \item $\rho$ - The inverse depth from the vehicle frame to the goal frame.
    Also note that rotation does not affect this state.
  \item $\vect{v}_{g/I}^{g}$ - The velocity of the goal frame w.r.t. inertial
    frame, expressed in the goal frame. The motion of the goal is most easily
    defined in the goal frame.
  \item $\theta_I^g$ - The angle that represents the rotation from the inertial
    frame to the goal frame. Note that this is only one value as the goal frame
    can only yaw while it is constrained to motion on the plane.
  \item $\omega_{g/I}^g$ - The angular rate of the goal frame w.r.t. inertial
    frame, expressed in the goal frame. This is the rate at which $\theta_I^g$
    changes.
  \item $\vect{r}_{i/g}^g$ - The location of landmark $i$ w.r.t. the goal frame,
    expressed in the goal frame. These landmarks are rigidly attached to the
    same landing vehicle as the goal frame, and therefore $\vect{r}_{i/g}^g$
    does not change, but is a constant.
  \item $\rho_i$ - The inverse depth from the vehicle frame to landmark $i$.
    This differs from $\rho$ because we cannot guarantee that all of the
    landmarks and the goal frame lie in the same plane.
\end{itemize}

\subsection{Goal State Dynamics}
We assume that the landing vehicle moves with a constant velocity and a constant
angular velocity. The dynamics of the goal states are expressed as
\begin{align}
  \dot{\vect{p}}_{g/v}^{v} &= \left(R_{v}^{g}\right)^\transpose
  \vect{v}_{g/I}^{g} - I_{2 \times 3}\left(R_{v}^{b}\right)^\transpose \vect{v}_{b/I}^{b} \\
  \dot{\rho} &= \rho^{2} \vect{e}_{3}^\transpose \left(R_{v}^{b}\right)^\transpose \vect{v}_{b/I}^{b} \\
  \dot{\vect{v}}_{g/I}^{g} &= \vect{0} + \eta_{v} \\
  \dot{\theta}_{I}^{g} &= \omega_{g/I}^g \\
  \dot{\omega}_{g/I}^{g} &= 0 + \eta_{\omega} \\
  \dot{\vect r}_{i}^{b} &= \vect{0} \\
  \dot{\rho_i} &= \rho_i^{2} \vect{e}_{3}^\transpose
    \left(R_{v}^{b}\right)^\transpose \vect{v}_{b/I}^{b}.
\end{align}.

Note that the rotation from the vehicle frame to the goal frame is given by
\begin{equation}
  R_v^g =
  \begin{bmatrix}
    \ctheta & \stheta \\
    -\stheta & \ctheta
  \end{bmatrix}.
\end{equation}

\subsection{Motion Model Jacobians}
The full state jacobian is given by
\begin{equation}
  A =
  \begin{bmatrix}
    \frac{\partial \dot{\x}_{\text{UAV}}}{\partial \x_{\text{UAV}}} &
    \frac{\partial \dot{\x}_{\text{UAV}}}{\partial \x_{\text{Goal}}} \\
    \frac{\partial \dot{\x}_{\text{Goal}}}{\partial \x_{\text{UAV}}} &
    \frac{\partial \dot{\x}_{\text{Goal}}}{\partial \x_{\text{Goal}}} 
  \end{bmatrix}.
\end{equation}
We have defined the first term, $\frac{\partial \dot{\x}_{\text{UAV}}}{\partial
\x_{\text{UAV}}}$ above in the previous section. We also note that
\begin{equation}
  \frac{\partial \dot{\x}_{\text{UAV}}}{\partial \x_{\text{Goal}}} = \vect{0}
\end{equation}
as the UAV dynamics do not depend on the landing vehicle dynamics. We define the
remaining terms here below.

\subsubsection{Jacobian w.r.t. UAV State}
\begin{equation}
  \frac{\partial \dot{\x}_{\text{Goal}}}{\partial \x_{\text{UAV}}}
  =
  \begin{bmatrix}
    \vect{0} & \frac{\partial \dot{\vect{p}}_{\text{Goal}}}{\partial
      \vect{\theta}_{\text{UAV}}} & \frac{\partial
      \dot{\vect{p}}_{\text{Goal}}}{\partial \vect{v}_{\text{UAV}}} & 0 \\
    \vect{0} & \frac{\partial \dot{\rho}_{\text{Goal}}}{\partial
      \vect{\theta}_{\text{UAV}}} & \frac{\partial
      \dot{\rho}_{\text{Goal}}}{\partial \vect{v}_{\text{UAV}}} & 0 \\
      \vect{0} & 0 & \vect{0} & 0 \\
    \vect{0} & \vect{0} & \vect{0} & 0 \\
    \vect{0} & \vect{0} & \vect{0} & 0 \\
    \vect{0} & \vect{0} & \vect{0} & 0 \\
    \vect{0} & \frac{\partial \dot{\rho}_{i}}{\partial
      \vect{\theta}_{\text{UAV}}} & \frac{\partial
      \dot{\rho}_{i}}{\partial \vect{v}_{\text{UAV}}} & 0
  \end{bmatrix}
\end{equation}
\begin{align}
    \frac{\partial \dot{\vect{p}}_{\text{Goal}}}{\partial
      \vect{\theta}_{\text{UAV}}}
      &=
      - I_{2 \times 3} \frac{\partial}{\partial \vect{\theta}_{\text{UAV}}}
      \left(R_{v}^{b}\right)^\transpose \vect{v}_{b/I}^{b}
      \\
    \frac{\partial \dot{\vect{p}}_{\text{Goal}}}{\partial \vect{v}_{\text{UAV}}}
      &=
      - I_{2 \times 3}\left(R_{v}^{b}\right)^\transpose
      \\
    \frac{\partial \dot{\rho}_{\text{Goal}}}{\partial
      \vect{\theta}_{\text{UAV}}}
      &=
      \rho^{2} \vect{e}_{3}^\transpose
        \frac{\partial}{\partial \vect{\theta}_{\text{UAV}}}
        \left(R_{v}^{b}\right)^\transpose \vect{v}_{b/I}^{b}
      \\
    \frac{\partial \dot{\rho}_{\text{Goal}}}{\partial \vect{v}_{\text{UAV}}}
      &=
      \rho^{2} \vect{e}_{3}^\transpose \left(R_{v}^{b}\right)^\transpose
      \\
    \frac{\partial \dot{\rho}_i}{\partial
      \vect{\theta}_{\text{UAV}}}
      &=
      \rho_i^{2} \vect{e}_{3}^\transpose
        \frac{\partial}{\partial \vect{\theta}_{\text{UAV}}}
        \left(R_{v}^{b}\right)^\transpose \vect{v}_{b/I}^{b}
      \\
    \frac{\partial \dot{\rho}_i}{\partial \vect{v}_{\text{UAV}}}
      &=
      \rho_i^{2} \vect{e}_{3}^\transpose \left(R_{v}^{b}\right)^\transpose
      \\
\end{align}

\subsubsection{Jacobian w.r.t. Goal State}
\begin{equation}
  \frac{\partial \dot{\x}_{\text{Goal}}}{\partial \x_{\text{Goal}}}
  =
  \begin{bmatrix}
    \vect{0} & 0 & \frac{\partial \dot{\vect{p}}_{\text{Goal}}}{\partial
      \vect{v}_{\text{Goal}}} & \frac{\partial
      \dot{\vect{p}}_{\text{Goal}}}{\partial \theta_{\text{Goal}}} & 0 & \vect{0} & 0 \\
    \vect{0} & \frac{\partial \dot{\rho}}{\partial \rho} & \vect{0} & 0 & 0
             & \vect{0} & 0 \\
    \vect{0} & 0 & \vect{0} & 0 & 0 & \vect{0} & 0 \\
    \vect{0} & 0 & \vect{0} & 0 & \frac{\partial
      \dot{\theta}_{\text{Goal}}}{\partial \omega_{Goal}} & \vect{0} & 0 \\
    \vect{0} & 0 & \vect{0} & 0 & 0 & \vect{0} & 0 \\
    \vect{0} & 0 & \vect{0} & 0 & 0 & \vect{0} & 0 \\
    \vect{0} & \frac{\partial \dot{\rho}_i}{\partial \rho_i} & \vect{0} & 0 & 0 & \vect{0} & 0
  \end{bmatrix}
\end{equation}
\begin{align}
    \frac{\partial \dot{\vect{p}}_{\text{Goal}}}{\partial
      \vect{v}_{\text{Goal}}}
      &=
      \left(R_{v}^{g}\right)^\transpose
      \\
    \frac{\partial \dot{\vect{p}}_{\text{Goal}}}{\partial \theta_{\text{Goal}}}
      &=
      \frac{\partial}{\partial \theta_{\text{Goal}}} \left(R_{v}^{g}\right)^\transpose \vect{v}_{g/I}^{g}
      \\
    \frac{\partial \dot{\rho}}{\partial \rho}
      &=
      2 \rho \vect{e}_{3}^\transpose \left(R_{v}^{b}\right)^\transpose \vect{v}_{b/I}^{b}
      \\
    \frac{\partial \dot{\theta}_{\text{Goal}}}{\partial \omega_{Goal}}
      &=
      1
      \\
    \frac{\partial \dot{\rho}_i}{\partial \rho_i}
      &=
      2 \rho_i \vect{e}_{3}^\transpose \left(R_{v}^{b}\right)^\transpose \vect{v}_{b/I}^{b}
\end{align}
